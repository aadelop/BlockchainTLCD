\chapter{Introducción}
\label{capitulo1}

La tecnología \texti{blockchain} está mayormente asociada al sector financiero y ha saltado a la palestra gracias a la aparición de las criptomonedas. Sin embargo,  empieza a despertar el interés de otros sectores, los cuales buscan aprovechar los beneficios que un registro distribuido e inmutable puede proveerles. Integridad, transparencia y confiabilidad son algunas de las características principales que hacen que las organizaciones, empresas y academias volteen su mirada hacia ella.

El presente trabajo busca exponer las características de la tecnología  \texti{blockchain} y cómo estas pueden ser utilizadas en beneficio de organizaciones que no necesariamente deben relacionarse al sector financiero. Para lograr un mayor entendimiento de la tecnología, se presentan conceptos teóricos fundamentales asociados a ella, muchos de los cuales no son nuevos, sino más bien heredados y adaptados de otras ramas de estudio. 

Ejemplo de esto es el registro distribuido (elemento fundamental del \texti{blockchain}), que proviene de la misma rama de estudio de los sistemas distribuidos. Por otro lado, de la gobernabilidad nace el concepto de descentralización (característica inherente de la tecnología que defiende la división equitativa del control o poder de una plataforma) y, finalmente, de la criptografía se toman conceptos y elementos que garantizan la seguridad del \texti{blockchain} a diferentes niveles (tal como las firmas digitales, los mecanismos de llaves criptográficas y las funciones hash).

Son varios los sectores y casos de uso donde la tecnología \texti{blockchain} puede ser aplicada de manera efectiva. Sin embargo, existen 2 formas de abordarla al ofrecer soluciones: los \texti{blockchain} públicos y los privados. Los primeros, están enfocados hacia aplicaciones descentralizadas donde la transparencia y trazabilidad son piezas claves para la organización o el negocio (la plataforma Ethereum plantea la solución más estable para este tipo de necesidades). En segundo lugar los \texti{blockchain} privados, orientados hacia  escenarios donde la privacidad e integridad son fundamentales dentro de los procesos internos de las organizaciones (la plataforma \texti{IBM Hyperledger}, actualmente es la más idónea para estos requerimientos).

El \texti{blockchain} va ganando terreno en diferentes sectores, empresas y naciones en todo el mundo al demostrar su poder y beneficios a lo largo del tiempo, principalmente en casos relacionados con dinero digital. Sin embargo, la variedad de usos que tiene esta tecnología han incrementado su incorporación en compañías, tanto en sus estructuras como en sus operaciones. 

Mejorar los sistemas de envíos, la legalización de documentos, rastrear diamantes y hasta promover educación de calidad, cuentan entre las infinitas aplicaciones de la cadena de bloques. Pero parece que el próximo ámbito donde esta tecnología pudiera ofrecer una verdadera revolución es en los centros educativos, en especial las universidades, donde sus beneficios pueden aplicarse tanto dentro como fuera del aula en cuanto a protección de datos, evitar el plagio, agilizar procesos de solicitud de información por parte de estudiantes y personal administrativo.


\section{Planteamiento del problema}

Los  sistemas distribuidos  son una rama de amplio estudio y trayectoria en el campo de la computación. Sin embargo, a medida que avanza la tecnología, la era digital y los requerimientos cambian, más específicamente, en los sistemas de bases datos, resulta necesario garantizar la seguridad, estabilidad y confiabilidad total de estos sistemas a gran escala, más aún aquellos cuya finalidad es manejar activos o recursos de alto riesgo, como por ejemplo el dinero, sin que dependan de una entidad de control única. 

Es aquí donde la descentralización entra en juego en forma de \textit{blockchain}, que a diferencia de las bases de datos distribuidas tradicionales, reparte el control y capacidad de verificación a todos los participantes involucrados (nodos), creando duplicados síncronos e idénticos de sus registros,  anulando los riesgos de corrupción de datos y permitiendo al libro contable ser inmutable. Por estas razones, el objetivo de este proyecto es estudiar las bases que fundamentan la teoría del \textit{blockchain} y explorar las distintas tecnologías y plataformas existentes para la realización de un prototipo funcional que simule el proceso de transcripción de programas de estudio inspirado en los procesos utilizados en la plataforma SIGPAE(Sistema de Gestión de Programas de Estudio) de la USB y así poder evaluar la factibilidad  y pertinencia de la  integración de características \textit{blockchain} dentro del sistema.


\section{Justificación e importancia}


Usando la tecnología, los participantes o actores tienen acceso a un libro de contabilidad digital compartido a través de una red de nodos o computadoras, prescindiendo de una autoridad central. Resulta importante resaltar que ningún involucrado tiene la capacidad de alterar los registros gracias a algoritmos matemáticos, los cuales garantizan su integridad. Son variados los rubros y casos de uso donde esta tecnología puede ser aplicada, brindado los siguientes beneficios:
\begin{itemize}
\item Intercambio sin intermediación de terceros. Se reduce gran parte del riesgo de confiar esas transacciones y operaciones a un tercero, o a un  reducido grupo de actores, lo que define la descentralización.
\item Garantía de plataformas transparentes e inmutables. Promueve la confianza de los participantes o actores sobre las operaciones y transacciones realizados en la misma.
\item Seguridad y verificación de identidades garantizada. Esto gracias a los  algoritmos criptográficos de gran complejidad y herramientas basadas en  firmas digitales.
\item Posibilidad de realizar o definir esquemas y reglas dentro de la plataforma, funcionando como contratos inteligentes. Esto aumenta el abanico de rubros y sectores donde esta tecnología puede aportar gran valor.
\item Flexibilidad para utilización de plataformas públicas (autónomas) o privadas, dependiendo de la necesidad y requerimientos del ecosistema, empresa o ente interesado en esta tecnología.
\end{itemize}

\section{Objetivo general de la investigación}

%\textcolor{blue}{CORRECION PROPUESTA: Afinación en objetivos específicos}
Estudiar los fundamentos teóricos de la tecnología \textit{blockchain}, para la posterior realización de una plataforma prototipo basada en esta tecnología y su aplicación a un caso de estudio de interés para la Universidad Simón Bolívar.

\subsection{Objetivos específicos}

\begin{enumerate}
  \item Estudiar los temas fundamentales que componen la teoría de la tecnología \textit{blockchain}: 
Sistemas Distribuidos, Conceptos básicos  \textit{blockchain}, Descentralización, Criptografía y Fundamentos Técnicos.

  \item Estudiar la distintas plataformas de tecnología \textit{blockchain} disponibles y confiables, tales como 
	IBM Hyperledger y Ethereum - Solidity.
  \item Desarrollo de funcionalidades y pequeños casos de uso para pruebas y exploración de las plataformas mencionadas.
  \item Desarrollo de un prototipo  en una de las plataformas \textit{blockchain} mencionadas, que simule el proceso interno de la de transcripción de programas de estudio de la plataforma SIGPAE de la Universidad Simón Bolívar.
\end{enumerate}

\section{Alcance de la investigación}

%\textcolor{blue}{CORRECION PROPUESTA: Afinación en alcance específicos}

El alcance de la investigación comprende desde  el estudio, exploración y profundización de los fundamentos teóricos asociados a la tecnología \textit{blockchain} hasta la realización de un prototipo de plataforma \textit{blockchain} funcional donde se reflejen todos los conceptos y fundamentos investigados, y que a la vez, se alinee con el objetivo de mejorar y optimizar los procesos administrativos internos dentro de la Universidad Simón Bolívar, específicamente y para este proyecto, procesos afines a la incorporación y gestión de programas analíticos de estudio.


\section{Estructura del trabajo}

%\textcolor{blue}{CORRECION PROPUESTA: Actualizacion de estructura final del trabajo}

A lo largo de este trabajo se explica en detalle los conceptos básicos y avanzados que componen toda la teoría del \textit{blockchain} necesaria para su entendimiento y uso potencial. De igual forma, se aborda el proceso de creación de un modelo de red \textit{blockchain}, necesario en toda plataforma o aplicación \textit{blockchain}, los detalles de implementación,  uso de las plataformas de la tecnologia y finalmente se expone la implementación de un modelo de red \textit{blockchain} real, que puede ser aplicado a procesos internos de la Universidad Simón Bolívar. En el Capitulo 1 se encuentre  el resumen e introducción, se explican los objetivos , justificación y alcance del proyecto de grado. El Capítulo 2 trata los temas teóricos de importancia para la comprensión de este trabajo. El Capítulo 3 se enfoca en el proceso utilizado en la producción de la plataforma \textit{blockchain}. El Capítulo 4 expone la implementación y características finales de la plataforma \textit{blockchain}. El Capítulo 5 expone las conclusiones y recomendaciones relacionadas al desarrollo y utilización de tecnologías \textit{blockchain} en el ámbito académico.


