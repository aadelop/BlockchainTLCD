% ************************** Thesis Abstract *****************************
% Use `abstract' as an option in the document class to print only the titlepage and the abstract
\begin{abstract}

La tecnología de libros contables distribuidos mejor conocida como \textit{blockchain}, tiene como objetivo, permitir el  registro distribuido e inmutable de transacciones entre entidades. Esta tecnología disruptiva ha empezado a tomar auge en los últimos años, más aún, en sectores o rubros que se distancian mucho de su uso o necesidad inicial (las criptomonedas). El campo educativo no es la excepción; no es difícil proyectar la cantidad de posibilidades que puede ofrecer una tecnología que garantice  inmutabilidad, verificación de identidades y auditoría en tiempo real de cualquier tipo de operación. Sus casos de uso se extienden desde cadena de suministros,  gestión estatal, sistema de votaciones y hasta el rastreo de encomiendas, solo por nombrar algunos de ellos.

%\textcolor{blue}{CORRECCIÓN SUGUERIDA: Afinación final del resumen. Hablar de sigpae}

Entendiendo el gran valor  que esta tecnología puede proveer, es menester de la academia aborse a su estudio e investigación. El presente proyecto, aunado a la creación del Grupo de Investigación Blockchain de la Universidad Simón Bolívar, tiene como finalidad dar inicio a la exploración e investigación de las distintas plataformas \textit{blockchain} existentes, populares y estables; evaluar casos de uso que sean de interés para la Universidad y, por consiguiente, para el grupo de investigación y posteriormente realizar un prototipo  funcional  que simula un proceso de transcripción de programas de estudio y cuyo objetivo es explorar características del \textit{blockchain} beneficiosas y pertinentes a incluir en el Sistema de Gestión de Programas  Analíticos de Estudios (SIGPAE) de la Universidad Simón Bolívar, que se encuentra actualmente en desarrollo  y sera responsable de la ejecución de todos los procesos necesarios para lograr la transcripción efectiva del contenido programático de una materia o curso y posteriormente lograr su inclusión dentro de los programas avalados. Finalmente en este contexto, en este documento se presenta la evaluación de 2 tecnologías \textit{blockchain} (\textit{Ethereum Platform} e \textit{IBM Hyperledger}),su comparación y  justificación  de selección  de \textit{IBM Hyperledger Fabric} para su uso en el desarrollo del prototipo funcional mencionado. Los procesos expuestos dan el punto de partida e inspiración para la demostración propuesta en este trabajo.

%\textcolor{red}{Escribir blockchain en todo el documento como \textit{blockchain}}

%\textcolor{red}{El resumen la vamos a revisar al final. Hay que mencionar el caso de uso que se modela}

\textbf{Palabras clave:} Cadenas de bloque, Contratos inteligentes, Criptografía, Libros Distribuidos, Descentralización.

\end{abstract}
