%!TEX root = ../thesis.tex

\section {Introducción y conceptos blockchain}  %Title of the First Chapter

 
\subsection {Sistemas distribuidos}

Primeramente es necesaria la  comprensión de los sistemas distribuidos porque básicamente la tecnología de blockchain  en su núcleo es un sistema distribuido, específicamente ,un sistema distribuido descentralizado.

Los sistemas distribuidos son un paradigma informático mediante el cual dos o más nodos trabajan entre sí en una manera coordinada para lograr un resultado común y está modelada de tal manera que los usuarios finales lo ven como una única plataforma lógica.\cite{bashir2017mastering}

Podemos entender a un nodo como un jugador individual en un sistema distribuido. Todos los nodos son capaces de enviar y recibir mensajes hacia y desde un nodo a otro. Los nodos pueden se pueden catalogar en 3 tipos honestos, defectuosos o maliciosos , poseen  memoria y procesador propios. Un nodo que puede mostrar un comportamiento arbitrario también se conoce como nodo bizantino. Este comportamiento arbitrario puede ser intencionalmente malicioso, lo cual es perjudicial para el funcionamiento del red. En general, cualquier comportamiento inesperado de un nodo en la red se puede categorizar como nodo bizantino.

El principal desafío que se presenta en los sistemas distribuidos es la coordinación entre nodos y tolerancia a fallas. Si algunos de los nodos se vuelven defectuosos o se rompen enlaces de red, el sistema distribuido debería ser capaz de  tolerar este tipo de situación y mantener su funcionamiento sin problemas para lograr el resultado deseado.


\subsection{Consenso}
El consenso es un proceso de acuerdo entre nodos desconfiados sobre un estado final de datos. Para lograr el consenso, se pueden usar diferentes algoritmos. En general es fácil llegar a un acuerdo entre dos nodos (por ejemplo en esquemas cliente-servidor) pero cuando múltiples nodos participan en un sistema distribuido y necesitan ponerse de acuerdo en un solo valor, se vuelve complejo lograr dicho consenso. El concepto que define la obtención de consenso entre nodos múltiples se conoce como consenso distribuido\cite{bashir2017mastering}

\textbf{Mecanismos de consenso:}  Un mecanismo de consenso es un conjunto de pasos que toman todos, o la mayoría de los nodos, para acordar un estado o valor propuesto\cite{swanson2015consensus}.Los mecanismos de consenso han pasado recientemente a ser el centro de atención y ganó mucha popularidad con el advenimiento de la primera implementación pública de blockchain la famosa plataforma Bitcoin.

Existen 5 requisitos que deben cumplirse para proporcionar los resultados deseados en un mecanismo de consenso:

\begin{itemize}
\item Acuerdo: todos los nodos confiables deciden sobre el mismo valor.
\item Terminación: todos los nodos confiables terminan la ejecución del proceso de consenso y eventualmente llegan a un decisión.
\item Validez: el valor acordado por todos los nodos honestos debe ser el mismo que el valor inicial propuesto por al menos un nodo confiable.
\item Tolerante a fallas: el algoritmo de consenso debería poder ejecutarse en presencia de fallas o malicia nodos (nodos bizantinos).
\item Integridad: este es un requisito donde ningún nodo toma la decisión más de una vez. Los nodos toman decisiones solo una vez en un solo ciclo de consenso.
\end{itemize}

Existen varios tipos de mecanismo de consenso, dentro de los 2 mas comunes encontramos:
\begin{itemize}
\item Basados en la tolerancia a fallas bizantinas: sin operaciones de cálculo intensivo, como la inversión de hash parcial, este método se basa en un esquema simple de nodos que publican mensajes firmados. Eventualmente, cuando se recibe una cierta cantidad de mensajes, se llega a un acuerdo. Dentro de las implementaciones más famosas de este tipo encontramos el  algoritmo Paxos publicado por Leslie Lamport \cite{lamport2001paxos}

\item Basados en líderes: este tipo de mecanismo requiere que los nodos compiten por la lotería de elección de líderes y el nodo que los gana proponga un valor final. Dentro de las implementaciones conocidas de este tipo encontramos al protocolo RAFT publicado por Diego Ongaro y John Ousterhout.\cite{ongaro2015raft}
\end{itemize}

El concepto de consenso de sistemas distribuidos ha sido incorporado  en la tecnología blockchain para proporcionar un medio aceptación y validación única de la verdad por parte de todos los pares en la red blockchain. Dentro de los algoritmos de consenso más importantes que están disponibles hoy o están siendo investigados en el contexto de blockchain encontramos:

\begin{itemize}
\item Prueba de trabajo: este tipo de mecanismo de consenso se basa en la prueba del consumo de recursos computacionales suficientes  antes de proponer un valor para su aceptación por parte de la red. Este mecanismo es utilizado en la plataforma Bitcoin  y otras plataformas blockchain asociadas a criptomonedas.
\item Prueba de apuesta: este algoritmo funciona con la idea de que un nodo o usuario tiene una apuesta suficiente en el sistema; por ejemplo, el nodo ha invertido los recursos suficientes  para que cualquier intento malicioso supere los beneficios de realizar un ataque al sistema. Esta idea fue presentada por el proyecto de Peercoin y se encuentra en etapa de prueba  en el blockchain de Ethereum. Otro concepto importante en este tipo de algoritmo es la “edad de la moneda”, que es un derivado desde la cantidad de tiempo y la cantidad de monedas que no se han gastado. En este modelo, las posibilidades de proponer y firmar el siguiente bloque aumentan con la edad de la moneda.
\item Prueba de importancia: esta prueba comparte similitud con la “Prueba de Apuesta” sin embargo este mecanismo no solo se basa en la cantidad de  apuesta que tiene un usuario en el sistema, sino que adicionalmente monitorea el uso y movimiento de los tokens por parte del usuario para establecer un nivel de confianza e importancia. Esto mecanismo es utilizado en el proyecto Nemcoin.
\item Prueba de tiempo transcurrido: introducido por Intel, usa un Entorno Confiable de Ejecución (TEE por sus siglas en inglés) para proporcionar aleatoriedad y seguridad en el proceso de elección del líder a través de un tiempo de espera garantizado. Requiere el Intel SGX (Software Guard Extension) procesador para proporcionar la garantía de seguridad\cite{costan2016intel}.
\item Consenso federado o consenso bizantino federado: los nodos en este protocolo mantienen un grupo de pares y compañeros de confianza pública propaga sólo aquellas transacciones que han sido validadas por la mayoría de los nodos de confianza. Usados en el protocolo de consenso del proyecto Stellar.
\item Mecanismos basados  reputación: mecanismo donde un líder es elegido sobre la base de la reputación que ha construido con el tiempo en el red. Esto puede basarse en la votación de otros miembros.
\end{itemize}


\subsection{Blockchain}

Desde una perspectiva técnica, podemos definir el blockchain  como un libro contable distribuido de tipo “punto-a-punto” (P2P) que es criptográficamente seguro, con política de solo-agregación, inmutable (extremadamente difícil de modificar), y actualizable solo por consenso o acuerdo entre pares. Adicionalmente podemos  considerar el blockchain  como una capa de red distribuida punto-a-punto que se ejecuta en la parte superior de Internet siendo análogo a como ocurre con los protocolos  SMTP, HTTP o FTP ejecutándose sobre TCP / IP.\cite{bashir2017mastering}

Desde el punto de vista de negocios,el blockchain se puede definir como una plataforma mediante la cual los pares pueden intercambiar valores usando transacciones sin la necesidad de un árbitro central de confianza. Esto permite que blockchain sea un mecanismo de consenso descentralizado donde ninguna autoridad única está a cargo de la base de datos.

Dentro de los elementos generales que debería contener cualquier plataforma blockchain se encuentran:
\begin{itemize}
    \item Direcciones: las direcciones son identificadores únicos que se utilizan en una transacción dentro del blockchain para designar remitentes y destinatarios. Una dirección generalmente es una clave pública . Si bien las direcciones pueden ser reutilizados por el mismo usuario, las direcciones son únicas. En la práctica, sin embargo, un solo usuario no puede usar la misma dirección nuevamente y generar una nueva para cada transacción. 
    \item Transacción:Una transacción es la unidad fundamental del blockchain. Una transacción representa una transferencia de valor de una dirección a otra.
    \item Bloque: Un bloque se compone de varias transacciones y otros elementos como el hash del bloque anterior (hash pointer) , timestamp y flags.
    \item Red punto-a-punto(P2P): esta es una topología de red mediante la cual todos los compañeros se pueden comunicar entre sí y enviar y recibir mensajes
    \item Scripting o lenguaje de programación: los scripts de transacción son conjuntos de comandos predefinidos para que los nodos transfieran tokens de una dirección a otra y realicen otras funciones.Una característica deseable de los lenguajes utilizados dentro del blockchain es que sean “Turing completos”.
    \item Contratos Inteligentes: programas que se ejecutan sobre el blockchain y encapsulan la lógica de negocios que se ejecutará cuando se cumplan ciertas condiciones. La función de contrato inteligente no está disponible en todas las cadenas de bloques, pero ahora se está convirtiendo en una característica muy deseable debido a la flexibilidad y la potencia que proporciona a las aplicaciones blockchain.
    \item Máquina Virtual: Una máquina virtual permite que el código Turing completo se ejecute en el blockchain (como contratos inteligentes), a diferencia  de un script de transacción que puede estar limitado en su funcionamiento. Las máquinas virtuales no están disponibles en todas las plataformas  blockchains; sin embargo, varios plataformas actuales usan máquinas virtuales para ejecutar programas, por ejemplo, Ethereum Virtual Machine (EVM) y Chain Virtual Machine (CVM).
    \item Nodos: Un nodo en una red blockchain realiza varias funciones dependiendo del rol que desempeña. Un nodo puede proponer y validar transacciones y realizar operaciones de minería para facilitar el consenso y asegurar la plataforma blockchain. Esto se hace siguiendo un protocolo o algoritmos de consenso explicados anteriormente . Los nodos también pueden realizar otras funciones, como la verificación simple de pagos (nodos livianos), validadores y muchos otras funciones según el tipo de blockchain utilizada y el rol asignado al nodo.
\end{itemize}

\textbf{Caracteristicas Principales:}

\begin{itemize}
    \item Consenso Distribuido: El consenso distribuido es el principal soporte de una plataforma blockchain. Esto permite que el blockchain presente una única versión de la verdad acordada por todas las partes sin el requisito de una autoridad central.
    \item Verificación de transacciones: Todas las transacciones publicadas por los  nodos en el blockchain se verifican en base a un conjunto predeterminado de reglas y solo se seleccionan las transacciones válidas para su inclusión en un bloque.
    \item Transferencia de valores entre pares: El Blockchain permite la transferencia de valor entre sus usuarios a través de tokens. Se puede pensar que los tokens son portadores de valor.
    \item Generación de criptomonedas: En un blockchain se  puede generar criptomonedas como un incentivo para sus mineros que validan las transacciones y gastan recursos para asegurar la plataforma. Esta es una función opcional según el tipo de blockchain utilizado.
    \item Propiedad intelectual: Es posible vincular un elemento digital o físico al blockchain de una manera irrevocable, de modo que no pueda ser reclamado por nadie más; el propietario tiene el control total de su activo y no puede ser gastado doblemente o tener doble propiedad. Esta característica tiene implicaciones de gran alcance, especialmente en la gestión de derechos digitales y en los sistemas electrónicos de efectivo, donde la detección de doble gasto es un requisito clave
    \item Proveedor de seguridad: El Blockchain se basa en una tecnología criptográfica comprobada que garantiza la integridad y la disponibilidad de los datos. Generalmente, la confidencialidad no se proporciona en primera instancia debido a los requisitos de transparencia, sin embargo la investigación en esta área ha  madurado y se ha avanzado mucho para obtener confidencialidad y privacidad en las plataformas blockchain que necesiten de estas características.
    \item Inmutabilidad: Otra característica clave de blockchain, los registros que se agregan una vez a la plataforma son inmutables. Aunque teóricamente existe la posibilidad de revertir los cambios, en la práctica se considera casi imposible de lograr, ya que se requiere de una cantidad inaccesible de recursos informáticos.
    \item Unicidad: Esta característica del blockchain garantiza que cada transacción sea única y no se haya consumido anteriormente. Esto es especialmente relevante en las plataformas blockchain de criptomonedas donde la detección y  evasión  del “doble gasto” son un requisito clave.
\end{itemize}

\textbf{Tipos de plataformas blockchain:} Basado en la evolución del blockchain en los últimos años, se puede dividir en varios tipos con atributos distintos que en muchos casos suelen solaparse.
\begin{itemize}
    \item Públicos:  Estos tipos de blockchains están abiertas al público y cualquiera puede participar como un nodo en el proceso de toma de decisiones y estos  pueden o no ser recompensados por su participación. Los registros no son propiedad de nadie y están públicamente abiertos para que cualquier persona participe y consulte. Todos los usuarios  mantienen una copia del blockchain en sus nodos locales y utilizan un mecanismo de consenso distribuido para tomar una decisión sobre el eventual estado del mismo.Este tipo  también se conocen como libros no permisados.
    \item Privados: Como su nombre lo indica son privados y están abiertos solo a un consorcio o grupo de individuos u organizaciones que han decidido compartir el libro de contabilidad entre ellos.
    \item Semi-privados: En este tipo de blockchains, una parte del mismo es privado y la otra es pública. La parte privada está controlada por un grupo de individuos, mientras que la parte pública está abierta para la participación de cualquier persona.
    \item Libros contable autorizado: Un libro de contabilidad autorizado es un blockchain por el cual los participantes de la red son conocidos y de confianza. Los libros mayores autorizados no necesitan usar un mecanismo de consenso distribuido, en su lugar se puede usar un protocolo de acuerdo para mantener una versión compartida de la verdad sobre el estado de los registros en el blockchain. Tampoco es obligatorio que una plataforma blockchain autorizada sea privada, ya que puede ser un blockchain público, pero con control de acceso regulado.
    \item Libros contable distribuido: Este libro se distribuye entre sus participantes y se extiende a través de múltiples sitios u organizaciones. Este tipo puede ser privado o público. La idea clave es que, a diferencia de muchas otras plataformas blockchain, los registros se almacenan de forma contigua en lugar de ordenarse en bloques. Ripple  es una de las plataforma  que ejemplifica este  tipo y la cual se explica más adelante en este proyecto.
    \item Tokenisadas: Son blockchains estándar que generan criptomonedas como resultado de un proceso de consenso a través de la minería o mediante la distribución inicial. Aquí entra cualquier plataforma relacionada a criptomonedas.
\end{itemize}

